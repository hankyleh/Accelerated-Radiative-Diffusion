\documentclass{template}
\title{Synthetic Acceleration for Radiative Diffusion Calculations\\
  \large Notes on Morel, Larsen, Matzen 1985~\cite{morel1985synthetic}}

\author{Kyle Hansen}
\date{10 October 2025}



\newcommand*\Let[2]{\State{#1 $\gets$ #2}}
\algrenewcommand\algorithmicrequire{\textbf{Precondition:}}
\algrenewcommand\algorithmicensure{\textbf{Postcondition:}}


\newcommand*\multbox[1]{\fbox{\hspace{0ex}#1\hspace{0ex}}}



\begin{document}

\maketitle

\section{Radiative Diffusion Equations}

The radiative diffusion equations are:

\begin{gather}
\label{eq:spectral_temperature}
\rho C_v \frac{\partial T}{\partial t} = \nabla \cdot K \nabla T - \nabla {\mathbb{P}u} + \rho \int_0^\infty \kappa(E^\prime)\left[ I(E^\prime) - \beta(E^\prime, T) \right] dE^\prime + Q\\
\label{eq:spectral_diffusion}
\frac{1}{c}\frac{\partial I}{\partial t} - \nabla \cdot D(E) \nabla I(E) + \rho \kappa(E)I(E) = \rho\kappa(E)\beta(E, T), 
\end{gather}

where $t$ is time, $\vec{r}$ is position, $E = h\nu$ is photon energy, $C_v$ is material specific heat, $K$ is thermal conductivity, $\mathbb{P}$ is the pressure tensor (from both material and radiation pressure), $u$ is material velocity, $\kappa$ is specific absorption opacity in units of $[\text{area}/\text{mass}]$, $\rho$ is material density, $D$ is the photon diffusion coefficient, $T$ is material temperature, $I(\vec{r}, E) = \int d\Omega \; I(x, \Omega, E)$ is photon intensity in units of $[\text{energy}/\text{area}\cdot\text{time}]$, and $Q$ is material energy source.

Integrating \autoref{eq:spectral_temperature} and \autoref{eq:spectral_diffusion} over $N_g$ energy groups yields the multigroup radiative diffusion equations:

\begin{gather}
  \label{eq:multigroup_temperature}
  \rho C_v \frac{\partial T}{\partial t} = \nabla \cdot K \nabla T - \nabla {\mathbb{P}u} + \rho \sum_{g=1}^{N_g} \kappa_g \left[ I_g - \beta_g(T) \right] + Q\\
  \label{eq:multigroup_diffusion}
  \frac{1}{c}\frac{\partial I_k}{\partial t} - \nabla \cdot D_k \nabla I_k + \rho \kappa_k I_k = \rho\kappa_k \beta_k(T); \qquad k=1\dots N_g
\end{gather}

where

\begin{subequations}
  \begin{align}
    \int_k \cdot \; dE &= \int_{E_k}^{E_{k+1}}\cdot\; dE\\
    I_k &= \int_k I(E) dE\\
    \beta_k(T) &= \int_k \beta(E, T) dE\\
    D_k \nabla I_k &= \int_k D(E) \nabla I(E) dE\\
    \kappa_k &= \frac{\int_g \kappa(E) \left[ \beta(E) - I(E) \right] dE}{\int_g \left[ \beta(E) - I(E) \right] dE}
  \end{align}
\end{subequations}


where the multigroup equations are only exact if the $E$-spectrum is known exactly within each energy group, though the approximation is close to exact when the number of energy groups is high. Rosseland- or Planck-weighted group constants are used to approximate the spectrum. Pressure $\mathbb{P}$, material velocity $\vec{u}$, thermal conductivity $K$, and density $\rho$ are taken to be known exactly, since they come from hydrodynamic equations.

Discretizing in time with Backward Euler and using lagged opacity, thermal conductivity, diffusion coefficient, and heat capacity:

\begin{gather}
  \rho C_v^n \frac{T^{n+1} - T^n}{\Delta t^n} = \nabla \cdot K^n \nabla T^{n+1} - \nabla {\mathbb{P}u} + \rho \sum_{g=1}^{N_g} \kappa_g^n \left[ I_g^{n+1} - \beta_g^{n+1}(T) \right] + Q^{n+1}\\
  \frac{1}{c}\frac{I^{n+1} - I^n}{\Delta t^n} - \nabla \cdot D_k^n \nabla I_k^{n+1} + \rho \kappa_k^n I_k^{n+1} = \rho \kappa_k^n \beta_k(T(t)); \qquad k=1\dots N_g
\end{gather}

Then, the time advancement of $\beta$ can be approximated using a first-order Taylor expansion, $\beta_k^{n+1} \approx \beta_k^n + \frac{\partial \beta_k^n}{\partial T}\Delta T^{n+1}$, and the temperature operator is split into two parts-- radiative/inhomogenous source and thermal conductivity/pressure. The linearized, time-discretized equations are:


\begin{gather}
    \label{eq:MEB}\rho C_v^n \frac{\Delta T^{n+1/2}}{\Delta t^n} =  \rho \sum_{g=1}^{N_g} \kappa_g^n \left[ I_g^{n+1} - \left( \beta_k^n + \frac{\partial \beta_k^n}{\partial T}\Delta T^{n+1}\right) \right] + Q^{n+1}\\
    \label{eq:intermediate_intensity}\frac{1}{c}\frac{I^{n+1} - I^n}{\Delta t^n} - \nabla \cdot D_k^n \nabla I_k^{n+1} + \rho \kappa_k^n I_k^{n+1} = \rho \kappa_k^n \left( \beta_k^n + \frac{\partial \beta_k^n}{\partial T}\Delta T^{n+1}\right); \qquad k=1\dots N_g\\
    \rho C_v^n \frac{\Delta T^{n+1}}{\Delta t^n} = \nabla \cdot K^n \nabla T^{n+1} - \nabla \cdot {\mathbb{P}u}
\end{gather}




Where $\Delta T^{n+1/2}$ is an intermediate value for the temperature at $t^{n+1}$, not at an intermediate time step, and $T^{n+1/2} = T^n + \Delta T^{n+1/2}$ and $T^{n+1} = T^{n+1/2} + \Delta T^{n+1}$. $\Delta T^{n+1/2}$ can be easily evaluated algebraically from \autoref{eq:MEB}:

\begin{equation}
  \label{eq:intermediate_temperature}\Delta T^{n+1/2} = \left[    \sum_{g=1}^{N_g} \kappa_g^n \left( I_g^{n+1} -  \beta_k^n  \right) +  \frac{Q^{n+1}}{\rho}    \right] \cdot {\left[  \frac{ C_v^n}{\Delta t^n}  + \sum_{g=1}^{N_g}\kappa_g^n \frac{\partial \beta_g^n}{\partial T}\right]}^{-1}
\end{equation}

Evaluating \autoref{eq:intermediate_intensity} using \autoref{eq:intermediate_temperature} yields

% TODO
\begin{equation}
  \frac{1}{c \Delta t^n}\left( I^{n+1} - I^n \right)- \nabla \cdot D_k^n \nabla I_k^{n+1} + \rho \kappa_k^n I_k^{n+1} = \dots TODO
\end{equation}

which reduces to the final form of the multigroup diffusion equation,

\begin{equation}\label{eq:standard_diffusion}
  \frac{1}{c \Delta t^n}\left( I^{n+1} - I^n \right)- \nabla \cdot D_k^n \nabla I_k^{n+1} + \rho \kappa_k^n I_k^{n+1} = \eta \chi_k \rho \sum_{g=1}^{N_g}\kappa_g I_g^{n+1} + q_k
\end{equation}

where
\begin{subequations}
  \begin{align}
    \eta   &=  \left[\sum_{g=1}^{N_g}\kappa_g\frac{\partial \beta_g}{\partial T}\right] \cdot {\left[ \frac{C_v}{\Delta t^n} + \sum_{g=1}^{N_g}\kappa_g\frac{\partial \beta_g}{\partial T} \right]}^{-1} \\
    \chi_k &=  \left[ \kappa_k  \frac{\partial \beta_k}{\partial T}   \right] \cdot {\left[\sum_{g=1}^{N_g}\kappa_g\frac{\partial \beta_g}{\partial T}\right]}^{-1} \\
    q_k    &=  \rho \kappa_k \beta_k  +  \eta \chi_k \left[ Q - \rho \sum_{g=1}^{N_g}\kappa_g\frac{\partial \beta_g}{\partial T} \right]
  \end{align}
\end{subequations}


\autoref{eq:standard_diffusion} resembles the steady-state neutron diffusion equation, which can be made clear by further rearranging the terms:

\begin{equation}\label{eq:neutron_diffusion}
  -\nabla \cdot D_k \nabla I_k^{n+1} + \left( \sigma_a + \sigma_{f, k} \right)I_k^{n+1} = \eta \chi_k \sum_{g=1}^{N_g}\sigma_{f, g}I_g^{n+1} + S_k
\end{equation}

where

\begin{subequations}\label{eq:neutron_xs}
  \begin{align}
  \sigma_a     &= \frac{1}{c\Delta t^n}\\
  \sigma_{f, k} &= \rho \kappa_k\\
  S_k &= q_k + \frac{I_k^n}{c\Delta t^n}
\end{align}
\end{subequations}

with ``fission'' and ``absorption'' cross sections, and fission emission spectrum $\chi_k$. The fission term will be known as the reimission term.

\subsection{Iterative solution}\label{sec:iterative_solution}


\autoref{eq:neutron_diffusion} can be solved by source iteration, starting from some initial guess $I_{k}^{(0)}$ and using to compute the next iterate $I^{(1)}$, or using each iterate $I_{k}^{l}$ to compute $I_{k}^{l+1}$. Surpressing the $n+1$ index, the iterative equation is

\begin{equation}\label{eq:iterative_neutron}
  -\nabla \cdot D_k \nabla I_k^{l+1} + \left( \sigma_a + \sigma_{f, k} \right)I_k^{l+1} = \eta \chi_k \sum_{g=1}^{N_g}\sigma_{f, g}I_g^l + S_k
\end{equation}

\section{Acceleration Scheme}

The iterative method described in \autoref{sec:iterative_solution} can be written in operator notation as the solution to the linear system

\begin{equation}
  \mathcal{M}x = y,
\end{equation}

where the exact solution is denoted as $x^\star$. The solution is solved iteratively by splitting the operator $\mathcal{M} = \mathcal{A} - \mathcal{B}$, and $\mathcal{A}$ is easily inverted. The iterative method is then

\begin{align}
  \mathcal{A}x^{l+1} &= \mathcal{B}x^l + y\\
  x^{l+1} &=  \mathcal{A}^{-1} \mathcal{B}x^l + \mathcal{A}^{-1}y.
\end{align}

For the radiative diffusion equation (\autoref{eq:neutron_diffusion}), $\mathcal{A}$ is the streaming and absorption operator, and $\mathcal{B}$ is the reemission (``fission'') source. The error of the $(l+1)$th iterate is

\begin{equation}
  \epsilon^{l+1} = x^\star - x^{l+1}
\end{equation}

Then the exact solution can be found from the $(l+1)$th iterate using

\begin{subequations}
  \begin{align}
  \mathcal{B}x^l + y &= \mathcal{A}x^{l+1}\\
  y & = \mathcal{A}x^{l+1} - \mathcal{B}x^l\\
  \mathcal{M}x^\star &= \mathcal{A}x^{l+1} - \mathcal{B}x^l\\
  x^\star &= \mathcal{M}^{-1} \left(\mathcal{A}x^{l+1} - \mathcal{B}x^l \right)\\
          &= \mathcal{M}^{-1} \left(\mathcal{A}x^{l+1} - \mathcal{B}x^{l+1}+\mathcal{B}x^{l+1}- \mathcal{B}x^l \right)\\
          &= \mathcal{M}^{-1} \left(\mathcal{M}x^{l+1} +\mathcal{B}{\left( x^{l+1}- x^l \right) }\right)\\
          \label{eq:exact_iterative}x^\star &=  x^{l+1} + \mathcal{M}^{-1}\mathcal{B}{\left( x^{l+1}- x^l \right)} = x^{l+1} + \epsilon^{l+1}
  \end{align}
\end{subequations}

The iteration can be accelerated by evaluating \autoref{eq:exact_iterative} using an approximation $\mathcal{W} \approx \mathcal{M}$, which can be more easily inverted. Here $\mathcal{M}$ is called the high-order operator, and $\mathcal{W}$ is the low-order operator.

The accelerated iteration scheme is

\begin{align}
  x^{l+1/2} &=  \mathcal{A}^{-1} \mathcal{B}x^l + \mathcal{A}^{-1}y\\
  x^{l+1}   &=  x^{l+1/2} + \mathcal{W}^{-1}\mathcal{B}{\left( x^{l+1/2}- x^l \right)}
\end{align}

when solving the radiative diffusion equation, the high-order operator $\mathcal{M}$ is the system of multigroup diffusion equations. The acceleerated scheme presented in~\cite{morel1985synthetic} uses a one-group diffusion operator as the low-order operator $\mathcal{W}$.

Fourier analyis of the high-order equation shows that the slowest-converging mode is ${\omega(\lambda=0)}$, or

\begin{equation}
  \omega_{\max} = \omega(0) = \eta \sum_{g=1}^{N_g} \frac{\chi_g \sigma_{f,g}}{\sigma_a + \sigma_{f,g}}
\end{equation}

Where $\omega_{\max}$ approaches $1$ as the $\Delta t^n$ is increased. For small time steps, $\omega_{\max} \approx \eta$, which approaches unity for strongly coupled problems, where $C_v$ is small or $\kappa$ is large.

The most effective acceleration schemes should approximate the high-order equations accurately about $\lambda = 0$. This solution corresponds with an equilibrium solution, which will be the focus of the low-order equation.~\autoref{sec:fourier_analysis} details the derivation of this result.

The multigroup equation for the error of the $l$th iterate of \autoref{eq:iterative_neutron} is

\begin{equation}
  \label{eq:error-diffusion-equation}-\nabla \cdot D_k\nabla \epsilon_k^{l} + \left(\sigma_a + \sigma_{f, k}\right)\epsilon_k^{l} = \eta \chi_k \sum_{g=1}^{N_g} \sigma_{f, g} \left[ \epsilon_g^{l} +I_g^{l} - I_g^{l-1} \right]; \; k=1, 2,\dots, N_g
\end{equation}

In the equilibrium limit, where the gradient term becomes zero, this equation becomes

\begin{align}
  \left(\sigma_a + \sigma_{f, k} \right)\epsilon_k^l &= \eta\chi_k \sum_{g=1}^{N_G}\sigma_{f, g}\epsilon_g^l + \eta\chi_k \sum_{g=1}^{N_G}\sigma_{f, g}\left(I_g^l - I_g^{l-1} \right)\\
  \sigma_{t, k}\epsilon_k^l &= \eta\chi_k \left( \xi^l + r^l \right)\\
  \epsilon_k^l &= \frac{\eta\chi_k}{\sigma_{t, k}} \left( \xi^l + r^l \right)\label{eq:first-step-eq-spectrum}
\end{align}

where
\begin{align}
  \sigma_{t, k} &= \sigma_a + \sigma_{f, k}\\
  r^l &= \sum_{g=1}^{N_G}\sigma_{f, g}\left(I_g^l - I_g^{l-1} \right)\\
  \xi^l &= \sum_{g=1}^{N_G}\sigma_{f, g}\epsilon_g^l
\end{align}

then, to solve for $\xi^i$, \autoref{eq:first-step-eq-spectrum} is multiplied by $\sigma_{f, k}$ and summed over all $k$:

\begin{align}
  \sum_{k=1}^{N_g} \sigma_{f, k}\epsilon_k^l &= \sum_{k=1}^{N_g} \sigma_{f, k}\frac{\eta\chi_k}{\sigma_{t, k}} \left( \xi^l + r^l \right)\\
  \xi^l &= \eta \left( \xi^l + r^l \right) \sum_{k=1}^{N_g}\frac{ \chi_k \sigma_{f,k}}{\sigma_{t, k}}\\
  \xi^l &= \eta \left( \xi^l + r^l \right) \frac{\langle \sigma_f \rangle}{\langle \sigma_t \rangle}\label{eq:second-step-eq-spectrum}
\end{align}

where

\begin{align}
  \langle \sigma_f \rangle &= \langle \sigma_t \rangle \sum_{g=1}^{N_g} \frac{\chi_g \sigma_{f, g}}{\sigma{t, g}}\\
  \langle \sigma_t \rangle &= {\left[\sum_{g=1}^{N_g} \frac{\chi_g}{\sigma_{t, g}} \right]}^{-1}
\end{align}

then \autoref{eq:second-step-eq-spectrum} can be solved for $\xi^l$,

\begin{subequations}
  \begin{align}
  \xi^l &= \eta \left( \xi^l + r^l \right) \frac{\langle \sigma_f \rangle}{\langle \sigma_t \rangle}\\
  \frac{\xi^l\langle \sigma_t \rangle}{\eta \langle \sigma_f \rangle} &= \left( \xi^l + r^l \right)\\
  \xi^l &= \frac{r^l}{\frac{\langle \sigma_t \rangle}{\eta \langle \sigma_f \rangle} - 1}\\
  \xi^l &= \frac{\eta \langle \sigma_f \rangle r^l}{\langle \sigma_t \rangle - \eta \langle \sigma_f \rangle}
\end{align}
\end{subequations}


yielding the final equilibrium solution of the $l$th iterate error, 

\begin{subequations}
  \begin{align}
    \epsilon_k^l &= \frac{\eta\chi_k}{\sigma_{t, k}} \left( \xi^l + r^l \right)\\
    \sigma_{t, k} &= \sigma_a + \sigma_{f, k}\\
  r^l &= \sum_{g=1}^{N_G}\sigma_{f, g}\left(I_g^l - I_g^{l-1} \right)\\
  \xi^l &= \frac{\eta \langle \sigma_f \rangle r^l}{\langle \sigma_t \rangle - \eta \langle \sigma_f \rangle}\\
  \langle \sigma_f \rangle &= \langle \sigma_t \rangle \sum_{g=1}^{N_g} \frac{\chi_g \sigma_{f, g}}{\sigma{t, g}}\\
  \langle \sigma_t \rangle &= {\left[\sum_{g=1}^{N_g} \frac{\chi_g}{\sigma_{t, g}} \right]}^{-1}
  \end{align}
\end{subequations}

Then the specral shape (normalized) of this solution, where $E^l = \sum_{g=1}^{N_g}\epsilon_k^l$, is

\begin{subequations}
  \begin{align}
  \frac{\epsilon_k^l}{E^l} &= \frac{\frac{\eta\chi_k}{\sigma_{t, k}} \left( \xi^l + r^l \right)}{\sum_{g=1}^{N_g}\frac{\eta\chi_g}{\sigma_{t, g}} \left( \xi^l + r^l \right)}\\
  &= \frac{\cancel{\eta\left( \xi^l + r^l \right)}\frac{\chi_k}{\sigma_{t, k}} }{\cancel{\eta\left( \xi^l + r^l \right)}\sum_{g=1}^{N_g}\frac{\chi_g}{\sigma_{t, g}} }\\
  &= \frac{\frac{\chi_k}{\sigma_{t, k}} }{\sum_{g=1}^{N_g}\frac{\chi_g}{\sigma_{t, g}} }\\
  \frac{\epsilon_k^l}{E^l} &= \frac{\langle \sigma_t \rangle \chi_k}{\sigma_{t, k}}.\label{eq:error-spectral-shape}
\end{align}
\end{subequations}


Which is exact as the time step $\Delta t$ approaches infinity (in the equilibrium limit), and is equivalent to the spectral shape used to evaluate Rosseland group opacities. Then, evaluating \autoref{eq:error-diffusion-equation} using the assumed spectral shape,

\begin{align}
  -\nabla \cdot D_k\nabla E^l\frac{ \langle \sigma_t \rangle \chi_k}{\sigma_{t, k}} + \left(\sigma_a + \sigma_{f, k}\right)E^l\frac{ \langle \sigma_t \rangle \chi_k}{\sigma_{t, k}} =\eta \chi_k r^l + \eta \chi_k \sum_{g=1}^{N_g} \sigma_{f, g} E^l\frac{ \langle \sigma_t \rangle \chi_k}{\sigma_{t, k}}.
\end{align}

Expanding the diffusion term using the product rule,


  \begin{multline}
  -\nabla \cdot \left[D_k \frac{ \langle \sigma_t \rangle \chi_k}{\sigma_{t, k}} \nabla E^l     +    D_k E^l \nabla   \frac{ \langle \sigma_t \rangle \chi_k}{\sigma_{t, k}}     \right]  + \left(\sigma_a + \sigma_{f, k}\right)E^l\frac{ \langle \sigma_t \rangle \chi_k}{\sigma_{t, k}} =\\
  \eta \chi_k r^l + \eta \chi_k \sum_{g=1}^{N_g} \sigma_{f, g} E^l\frac{ \langle \sigma_t \rangle \chi_k}{\sigma_{t, k}}.
\end{multline}



Summing over all groups, 

\begin{multline}
  -\nabla \cdot \left[\left( \sum_{k=1}^{N_g}D_k \frac{ \langle \sigma_t \rangle \chi_k}{\sigma_{t, k}}\right) \nabla E^l     +   E^l \left( \sum_{k=1}^{N_g} D_k  \nabla   \frac{ \langle \sigma_t \rangle \chi_k}{\sigma_{t, k}} \right)    \right]  + E^l \sum_{k=1}^{N_g}\left(\sigma_a + \sigma_{f, k}\right)\frac{ \langle \sigma_t \rangle \chi_k}{\sigma_{t, k}} =\\
  \eta  r^l + \sum_{k=1}^{N_g}\eta E^l  \chi_k  \langle \sigma_t \rangle \sum_{g=1}^{N_g} \sigma_{f, g} \frac{  \chi_g}{\sigma_{t, g}}
\end{multline}

\begin{multline}
  -\nabla \cdot \left[\left( \sum_{k=1}^{N_g}D_k \frac{ \langle \sigma_t \rangle \chi_k}{\sigma_{t, k}}\right) \nabla E^l     +   E^l \left( \sum_{k=1}^{N_g} D_k  \nabla   \frac{ \langle \sigma_t \rangle \chi_k}{\sigma_{t, k}} \right)    \right]  + E^l \sum_{k=1}^{N_g}\left(\sigma_a + \sigma_{f, k}\right)\frac{ \langle \sigma_t \rangle \chi_k}{\sigma_{t, k}} =\\
  \eta  r^l + \eta E^l   \langle \sigma_f \rangle,
\end{multline}

which can be written as

\begin{equation}
  -\nabla \cdot \left[\langle D \rangle \nabla E^l     +  \langle D^\prime \rangle  E^l    \right]  + E^l \sum_{k=1}^{N_g}\left(\sigma_a + \sigma_{f, k}\right)\frac{ \langle \sigma_t \rangle \chi_k}{\sigma_{t, k}} =  \eta  r^l + \eta E^l   \langle \sigma_f \rangle,
\end{equation}

where

\begin{subequations}
 \begin{align}
  \langle D \rangle &= \sum_{k=1}^{N_g}D_k \frac{ \langle \sigma_t \rangle \chi_k}{\sigma_{t, k}}  \\
  \langle D^\prime \rangle &= \sum_{k=1}^{N_g} D_k  \nabla   \frac{ \langle \sigma_t \rangle \chi_k}{\sigma_{t, k}}.
\end{align} 
\end{subequations}


Expanding the removal term, 

\begin{subequations}
  \begin{align}
  E^l \sum_{k=1}^{N_g}\left(\sigma_a + \sigma_{f, k}\right)\frac{ \langle \sigma_t \rangle \chi_k}{\sigma_{t, k}} & =  \sigma_a \sum_{k=1}^{N_g}E^l\frac{ \langle \sigma_t \rangle \chi_k}{\sigma_{t, k}} + \sum_{k=1}^{N_g} E^l\frac{ \langle \sigma_t \rangle \chi_k \sigma_{f, k}}{\sigma_{t, k}}\\
  &=  \sigma_a E^l + \langle\sigma_f \rangle E^l,
\end{align}
\end{subequations}


so the one-group error diffusion equation becomes

\begin{align}
  -\nabla \cdot \left[\langle D \rangle \nabla E^l     +  \langle D^\prime \rangle  E^l    \right]  + E^l (\sigma_a + \langle \sigma_f \rangle) &=  \eta  r^l + \eta E^l   \langle \sigma_f \rangle\\
  -\nabla \cdot \left[\langle D \rangle \nabla E^l     +  \langle D^\prime \rangle  E^l    \right]  + E^l \left[\sigma_a + \langle \sigma_f \rangle (1 - \eta)\right] &=  \eta  r^l. 
\end{align}

This provides an equation for the one-group iterative error of a diffusion iteration, which can be used to calculate group errors using \autoref{eq:error-spectral-shape}. Then the set of equations for the accelerated radiative diffusion equation becomes:

\begin{gather}
  \label{eq:diffusion-half-iter}-\nabla \cdot D_k \nabla I_k^{l+1} + \left( \sigma_a + \sigma_{f, k} \right)I_k^{l+1} = \eta \chi_k \sum_{g=1}^{N_g}\sigma_{f, g}I_g^l + S_k\\
  \label{eq:onegroup-error-iter}-\nabla \cdot \left[\langle D \rangle \nabla E^{l+1/2}      +  \langle D^\prime \rangle  E^{l+1/2}    \right]  + E^{l+1/2}  \left[\sigma_a + \langle \sigma_f \rangle (1 - \eta)\right] =  \eta  r^{l+1/2} \\
  \label{eq:fix-diffuion}I^{l+1} = I_k^{l+1} + E^{l+1/2}\frac{\langle \sigma_t \rangle \chi_k}{\sigma_{t, k}}
\end{gather}

where

\begin{equation}
  r^{l+1/2} = \sum_{g=1}^{N_g} \sigma{f, g} \left( I_g^{l+1/2} - I_g^l\right).
\end{equation}

The full algorithm for the time-dependent problem is shown in Algorithm~\ref{alg:packed-dna-hamming}.

\begin{algorithm}
  \caption{Morel, Larsen, Matzen Diffusion Acceleration Scheme~\cite{morel1985synthetic}}\label{alg:packed-dna-hamming}
  \begin{algorithmic}[0]
    % \Require{$x$ and $y$ are packed dna strings of equal length $n$}
    \Statex{}
    \While{$t^j \leq t^{end}$}
      \Let{$j$}{$j+1$}
      \State{Compute temperature-dependent coefficients using $T\vert_{t=t^{j-1}}$}
      \While{$\norm{I^{(s)} - I^{(s-1)}} > \epsilon\norm{I^{(s-1)}}$}
        \Let{$s$}{$s+1$}
        \State{Solve multigroup diffusion equation for $I_k^{(s+1/2)}$, using $I_k^{(s)}$ for the reemission source}
        \State{Compute $r^{(s+1/2)}$}
        \State{Solve one-group error diffusion equation for $E^{(s+1/2)}$}
        \State{Compute $I_k^{(s+1)} = I_k^{(s+1/2)} + E^{(s+1/2)}\langle \sigma_t \rangle \chi_k / \sigma_{t, k}$}
      \EndWhile{}
      \Let{$I_k^{j}$}{$I_k^{(s+1)}$}
      \State{Compute $\Delta T^{j+1/2}$ from $I_k^{j}$}
      \If{$K \neq 0 \land u \neq 0$}
        \State{Solve thermal diffusion equation for $\Delta T^{j+1}$}
      \EndIf{}
      \Let{$T^{j+1}$}{$T^j + \Delta T^{j+1/2} + \Delta T^{j+1}$}
    \EndWhile{}
  \end{algorithmic}
\end{algorithm}

\section{Fourier Analysis}\label{sec:fourier_analysis}

The iterative diffusion equation, \autoref{eq:iterative_neutron}, is


\begin{equation}
  -\nabla \cdot D_k \nabla I_k^{l+1} + \left( \sigma_a + \sigma_{f, k} \right)I_k^{l+1} = \eta \chi_k \sum_{g=1}^{N_g}\sigma_{f, g}I_g^l + S_k.
\end{equation}

Subtracting this from the exact equation, \autoref{eq:neutron_diffusion}, yields the iterative equation for error:

\begin{align}
  -\nabla \cdot D_k \nabla \left(I_k^\star - I_k^{l+1}\right) + \left( \sigma_a + \sigma_{f, k} \right)\left(I_k^\star - I_k^{l+1}\right) &= \eta \chi_k \left(\sum_{g=1}^{N_g}\sigma_{f, g}I_g^\star - \sum_{g=1}^{N_g}\sigma_{f, g}I_g^l\right)\\
  -\nabla \cdot D_k \nabla \epsilon_k^{l+1} + \left( \sigma_a + \sigma_{f, k} \right)\epsilon_k^{l+1} &= \eta \chi_k \sum_{g=1}^{N_g}\sigma_{f, g}\epsilon_g^{l}.\label{eq:iterative-error-convergence} 
\end{align}

For infinite 1D spatial domain and constant coefficients, the Fourier Ansatz assumes a form of the error (and solution),

\begin{equation}
  \epsilon_k^{l}(x) = \epsilon_k^l e^{i\lambda x}.
\end{equation}

Using this assumed form, \autoref{eq:iterative-error-convergence} is

\begin{align}
  -D_k \epsilon_k^{l+1}\nabla^2 e^{i\lambda x} + \left( \sigma_a + \sigma_{f, k} \right)\epsilon_k^{l+1} e^{i\lambda x} &= \eta \chi_k  e^{i\lambda x}\sum_{g=1}^{N_g}\sigma_{f, g}\epsilon_g^{l}\\
  e^{i\lambda x} \left[\lambda^2 D_k \epsilon_k^{l+1} + \left( \sigma_a + \sigma_{f, k} \right)\epsilon_k^{l+1} \right] &= e^{i\lambda x} \left[\eta \chi_k  \sum_{g=1}^{N_g}\sigma_{f, g}\epsilon_g^{l}\right]\\
  \epsilon_k^{l+1} \left[\lambda^2 D_k+ \sigma_a + \sigma_{f, k} \right] &= \eta \chi_k  \sum_{g=1}^{N_g}\sigma_{f, g}\epsilon_g^{l}\\
  \epsilon_k^{l+1}  &=  \frac{\eta \chi_k  \sum_{g=1}^{N_g}\sigma_{f, g}\epsilon_g^{l}}{\lambda^2 D_k+ \sigma_a + \sigma_{f, k}}.
\end{align}

Then, using $F^l = \sum_{k=1}^{N_g}\sigma_{f, k}\epsilon_k^l$, this equation becomes

\begin{align}
  \sum_{k=1}^{N_g}\sigma_{f, k}\epsilon_k^{l+1}  &= \sum_{k=1}^{N_g} \frac{\sigma_{f, k}\eta \chi_k  \sum_{g=1}^{N_g}\sigma_{f, g}\epsilon_g^{l}}{\lambda^2 D_k+ \sigma_a + \sigma_{f, k}}\\
  F^{l+1}  &= \sum_{k=1}^{N_g} \frac{\sigma_{f, k}\eta \chi_k  F^l}{\lambda^2 D_k+ \sigma_a + \sigma_{f, k}}\\
  F^{l+1}  &=  F^l\eta \sum_{k=1}^{N_g} \frac{\sigma_{f, k} \chi_k }{\lambda^2 D_k+ \sigma_a + \sigma_{f, k}},
\end{align}

which corresponds to a Source Iteration eigenvalue of 

\begin{equation}\label{eq:fourier-eigenvalue}
  \omega(\lambda) = \eta \sum_{k=1}^{N_g} \frac{\sigma_{f, k} \chi_k }{\lambda^2 D_k+ \sigma_a + \sigma_{f, k}},
\end{equation}

where 

\begin{equation}
  \sum_{k=1}^{N_g}\sigma_{f, k}\epsilon_k^{l+1} = \omega(\lambda) \sum_{k=1}^{N_g}\sigma_{f, k}\epsilon_k^l.
\end{equation}

The spectral radius corresponds with the eigenmode $\lambda=0$, and is equal to 

\begin{equation}
  \text{spr} = \eta \sum_{k=1}^{N_g} \frac{\sigma_{f, k} \chi_k }{\sigma_a + \sigma_{f, k}}
\end{equation}

As the time step $\Delta t$ becomes large, $\sigma_a \rightarrow 0$ and $\eta \rightarrow 1$, and from \autoref{eq:fourier-eigenvalue}, the spectral radius approaches one. $\eta$ also approaches one for large opacity or small heat capacity, or in strong radiation-material coupling.

\section{Numerical Implementation}

\autoref{eq:diffusion-half-iter}, \ref{eq:onegroup-error-iter}, and \ref{eq:fix-diffuion} are the iterative algorithm for radiative diffusion with implicit temporal differencing and lagged coefficients. If it is further assumed that the temperature is spatially invariant within each cell, then the constant $\langle D^\prime \rangle$ becomes zero and the equations are

\begin{gather}
  \label{eq:og-equation}-\nabla \cdot D_k \nabla I_k^{l+1} + \left( \sigma_a + \sigma_{f, k} \right)I_k^{l+1} = \eta \chi_k \sum_{g=1}^{N_g}\sigma_{f, g}I_g^l + S_k\\
  -\nabla \cdot \langle D \rangle \nabla E^{l+1/2}       + E^{l+1/2}  \left[\sigma_a + \langle \sigma_f \rangle (1 - \eta)\right] =  \eta  r^{l+1/2} \\
  I^{l+1} = I_k^{l+1} + E^{l+1/2}\frac{\langle \sigma_t \rangle \chi_k}{\sigma_{t, k}}.
\end{gather}

To solve using the Linear Discontinuous (LD) discretization, we first divide the domain $(0, X)$ into $N_x$ cells $K_i = \left[ x_{i-1/2},\;x_{i+1/2} \right], \qquad i=1,\hdots N$ with cell width $\Delta x_i = x_{i+1/2} - x_{i-1/2}$. Then let $I_{i}(x)$ represent the intensity in cell $i$, and $F_i$ represent the flux in cell $i$, where, from the diffuion approximation,

\begin{equation}
  F_i = -D_i \cdot \nabla I_i.
\end{equation}

Then, in one dimension only, the diffusion equation \ref{eq:og-equation} can be expressed (surpressing $k$ subscripts) as the system

\begin{subequations}\label{eq:p1-equations}
 \begin{align}
  \label{eq:zeroth-moment}\frac{d}{dx} F^{l+1}_i + \left( \sigma_a + \sigma_{f} \right)I^{l+1}_i &= \eta \chi \sum_{g=1}^{N_g}\sigma_{f, g}I_{i, g}^l + S_i\\
  \label{eq:first-moment}F_i &= -D_i \cdot \frac{d}{dx} I_i.
\end{align} 
\end{subequations}


In the LD discretization, the intensity in cell $K_i$ is

\begin{gather}
  I_i(x) = I_{L, i} b_{L, i}(x) +  I_{R, i} b_{R, i}(x),\\
  b_{L, i}(x) = \frac{x_{i+1/2}-x}{\Delta x_i}, \qquad b_{R, i}(x) = \frac{x - x_{i-1/2}}{\Delta x_i}
\end{gather}

then $I_{L, i}$ and $I_{R, i}$ are the unknown intensity values at the left and right edges of cell $i$. The cell-average intensity is given by

\begin{equation}
    \overline{I_i} = \frac{1}{2}\left(I_{L, i} +  I_{R, i}\right).
\end{equation}

The same discretization is applied to flux $F$. Multiplying \autoref{eq:zeroth-moment} by the test functions $b_c, c=L, R$ and integrating over the cell $K_i$ yields

\begin{multline}
  \int_{x_{i-1/2}}^{x_{i+1/2}}b_c(x) \frac{d}{dx} F^{l+1}_i(x) dx + \left( \sigma_a + \sigma_{f} \right) \int_{x_{i-1/2}}^{x_{i+1/2}}b_c(x) I^{l+1}_i dx=\\
   \eta \chi \sum_{g=1}^{N_g}\sigma_{f, g} \int_{x_{i-1/2}}^{x_{i+1/2}}b_c(x) I_{i, g}^l dx+ \int_{x_{i-1/2}}^{x_{i+1/2}}b_c(x)S_i(x) dx.
\end{multline}

Integrating the first term by parts yields

\begin{align}
  \left[ b_c (x) F_i (x) \right]_{x_{i-1/2}}^{x_{i+1/2}} &- \int_{x_{i-1/2}}^{x_{i+1/2}} \frac{db_c}{dx}F_i (x) dx\\
  \left[ b_c (x) F_i (x) \right]_{x_{i-1/2}}^{x_{i+1/2}} & \pm \frac{1}{\Delta x}\int_{x_{i-1/2}}^{x_{i+1/2}} F_i (x) dx\\
  \left[ b_c (x) F_i (x) \right]_{x_{i-1/2}}^{x_{i+1/2}} & \pm  \frac{1}{2}(F_L + F_R)
\end{align}

where one of the boundary terms vanishes for each value of $c$, and the sign of the second term depends again on $c$, since $db_L/dx = -1/\Delta x$ and $db_R/dx = 1/\Delta x$. Then, the $\int b I$ terms are expanded in terms of the mass matrix,

\begin{gather}
  \mat{M}_i = \frac{1}{6}\begin{bmatrix}
    m_{LL} & m_{LR}\\
    m_{RL} & m_{RR}
  \end{bmatrix}\\
  \mat{M}_{i, LD} = \frac{1}{6}\begin{bmatrix}
    2 & 1\\
    1 & 2
  \end{bmatrix}\\
  \mat{M}_{i, LLD} = \frac{1}{6}\begin{bmatrix}
    3 & 0\\
    0 & 3
  \end{bmatrix}.
\end{gather}


\autoref{eq:first-moment} is written in the same way as the first ($d/dx$) term of \autoref{eq:zeroth-moment}. Then, for $c=L$,

\begin{subequations}\label{eq:left}
  \begin{multline}
    -F^{b, l+1}(x_{i-1/2}) + \frac{1}{2} \left(F_{L, i}^{l+1} + F_{R, i}^{l+1}\right) + \left( \sigma_{a, i} + \sigma_{f, k, i} \right) \left( m_{LL}  I_{L, i}^{l+1} +m_{LR} I_{R, i}^{l+1}   \right)\frac{\Delta x_i}{6} =\\
    \eta_i \chi_{k, i} \sum_{g=1}^{N_g}\sigma_{f, g, i} \left( m_{LL}  I_{L, i, g}^{l} +m_{LR} I_{R, i, g}^{l}   \right)\frac{\Delta x_i}{6} + \left( m_{LL}  S_{L, i, k}^{l+1} +m_{LR} S_{R, i, k}^{l+1}   \right)\frac{\Delta x_i}{6}
  \end{multline}
  \begin{equation}
    \left( m_{LL}  F_{L, i, k}^{l+1} +m_{LR} F_{R, i, k}^{l+1}   \right)\frac{\Delta x_i}{6} = -D_{k, i} \left[ -I_k^{b, l+1}(x_{i-1/2}) +   \frac{1}{2} \left(I_{L, i, k}^{l+1} + I_{R, i, k}^{l+1}\right)\right],
  \end{equation}
\end{subequations}

and for $c=R$, 

\begin{subequations}\label{eq:right}
  \begin{multline}
    F^{b, l+1}(x_{i+1/2}) - \frac{1}{2} \left(F_{L, i}^{l+1} + F_{R, i}^{l+1}\right) + \left( \sigma_{a, i} + \sigma_{f, k, i} \right) \left( m_{RL}  I_{L, i}^{l+1} +m_{RR} I_{R, i}^{l+1}   \right) \frac{\Delta x_i}{6}=\\
    \eta_i \chi_{k, i} \sum_{g=1}^{N_g}\sigma_{f, g, i} \left( m_{RL}  I_{L, i, g}^{l} +m_{RR} I_{R, i, g}^{l}   \right) \frac{\Delta x_i}{6}+ \left( m_{RL}  S_{L, i, k}^{l+1} +m_{RR} S_{R, i, k}^{l+1}   \right)\frac{\Delta x_i}{6}
  \end{multline}
  \begin{equation}
    \left( m_{LL}  F_{L, i, k}^{l+1} +m_{LR} F_{R, i, k}^{l+1}   \right)\frac{\Delta x_i}{6} = -D_{k, i} \left[ I_k^{b, l+1}(x_{i+1/2}) -   \frac{1}{2} \left(I_{L, i, k}^{l+1} + I_{R, i, k}^{l+1}\right)\right].
  \end{equation}
\end{subequations}


Since the values of $I$ and $F$ are discontinuous on cell boundaries, the following interface conditions are used:

\begin{gather}
  F^b(x_{i+1/2}) = \frac{1}{2}\left(F_{R, i} - F_{L, i+1} \right) + \frac{1}{4}\left( I_{R, i} + I_{L, i+1} \right)\\
  I^b(x_{i+1/2}) = \frac{3}{4}\left( F_{R, i} - F_{L, i+1} \right) + \frac{1}{2}\left(I_{R, i} + I_{L, i+1} \right).
\end{gather}

The Marshak boundary conditions, which fix the interface conditions $F^b$ and $I^b$ at $x = 0, X$, become


  \begin{subequations}\label{eq:left-boundary}
    \begin{gather}
      I^b(0) = I^{in,+} + \frac{1}{2}\left( I_{L, 1} - \frac{3}{2}F_{L, 1} \right)\\
      F^b(0) = F^{in,+} + \frac{1}{4}I_{L, 1} - \frac{1}{2}F_{L, 1}
  \end{gather}
  \end{subequations}
  \begin{subequations}\label{eq:right-boundary}
    \begin{gather}
      I^b(X) = I^{in,-} + \frac{1}{2}\left( I_{R, N_x} + \frac{3}{2}F_{R, N_x} \right)\\
      F^b(X) = F^{in,-} + \frac{1}{4}I_{R, N_x} + \frac{1}{2}F_{R, N_x},
    \end{gather}
  \end{subequations}

where

\begin{gather}
  F^{in, -} = \int_{-1}^{0} \mu I^{in, -}(X, \mu)d \mu, \qquad F^{in, +} = \int_{0}^{+1} \mu I^{in, +}(0, \mu)d \mu\\
  I^{in, -} = \int_{-1}^{0}  I^{in, -}(X, \mu)d \mu, \qquad I^{in, +} = \int_{0}^{+1}  I^{in, +}(0, \mu)d \mu\\
\end{gather}

Then, for the interior cells, from $\int_{K_i} \cdot \; b_L dx$ we obtain  

\begin{subequations}\label{eq:left-equations}
  \begin{multline}
    \left[-\frac{1}{4} \left( I_{R, i-1}^{n+1} + I_{L, i}^{n+1}\right) + \left(\sigma_a^n + \sigma_{f,k}^n \right) \left(m_{LL}I_{L, i}^{n+1} + m_{LR}I_{R, i}^{n+1}\right)\frac{\Delta x_i}{6}\right] \\
    + \left[ -\frac{1}{2}\left( F_{R, i-1}^{n+1} - F_{L,i}^{n+1}\right) + \frac{1}{2}\left( F_{L, i}^{n+1} + F_{R, i}^{n+1}\right)\right] =\\
    \left[\eta \chi_k \sum_{g=1}^{N_g}\sigma_{f, g}\left(m_{LL}I_{L, i, g}^n + m_{LR}I_{R, i, g}^n\right)\frac{\Delta x_i}{6} + \frac{1}{c\Delta t^n}\left(m_{LL}I_{L, i}^n + m_{LR}I_{R, i}^n\right)\frac{\Delta x_i}{6} \right]\\
    + \left[\left(m_{LL}q_{L, i} + m_{LR}q_{R, i}\right)\frac{\Delta x_i}{6}\right]
  \end{multline}
  \begin{multline}
    \left[\left(m_{LL}F_{L, i}^{n+1} + m_{LR}F_{R, i}^{n+1}\right)\frac{\Delta x_i}{6} -\frac{3}{4}  D_k \left( F_{R, i-1}^{n+1} - F_{L, i}^{n+1} \right)\right]\\
    +\left[- \frac{1}{2}D_k \left( I_{R, i-1}^{n+1} -+I_{L, i}^{n+1}\right)  + \frac{1}{2}D_k \left(I_{L, i}^{n+1} + I_{R, i} ^{n+1}\right)\right]=0,
  \end{multline}
\end{subequations}

for $i=2,\hdots, N_x$, and from $\int_{K_i} \cdot \; b_R dx$, 


\begin{subequations}\label{eq:right-equations}
  \begin{multline}
    \left[\frac{1}{4} \left( I_{R, i}^{n+1} + I_{L, i+1}^{n+1}\right) + \left(\sigma_a^n + \sigma_{f,k}^n \right) \left(m_{RL}I_{L, i}^{n+1} + m_{RR}I_{R, i}^{n+1}\right)\frac{\Delta x_i}{6}\right] \\
    + \left[ \frac{1}{2}\left( F_{R, i}^{n+1} - F_{L,i+1}^{n+1}\right) - \frac{1}{2}\left( F_{L, i}^{n+1} + F_{R, i}^{n+1}\right)\right] =\\
    \left[\eta \chi_k \sum_{g=1}^{N_g}\sigma_{f, g}\left(m_{RL}I_{L, i, g}^n + m_{RR}I_{R, i, g}^n\right)\frac{\Delta x_i}{6} + \frac{1}{c\Delta t^n}\left(m_{RL}I_{L, i}^n + m_{RR}I_{R, i}^n\right)\frac{\Delta x_i}{6} \right]\\
    + \left[\left(m_{RL}q_{L, i} + m_{RR}q_{R, i}\right)\frac{\Delta x_i}{6}\right]
  \end{multline}
  \begin{multline}
    \left[\left(m_{RL}F_{L, i}^{n+1} + m_{RR}F_{R, i}^{n+1}\right)\frac{\Delta x_i}{6} + \frac{3}{4}  D_k \left( F_{R, i}^{n+1} - F_{L, i+1}^{n+1} \right)\right]\\
    +\left[\frac{1}{2}D_k \left( I_{R, i}^{n+1} + I_{L, i+1}^{n+1}\right)  - \frac{1}{2}D_k \left(I_{L, i}^{n+1} + I_{R, i} ^{n+1}\right)\right]=0.
  \end{multline}
\end{subequations}

for $i=1, \hdots, (N_x - 1)$. In the left boundary cell, using \autoref{eq:left-boundary}, we have

\begin{subequations}
  \begin{multline}
    \left[\frac{1}{4} \left( I_{L, 1}^{n+1}\right) + \left(\sigma_a^n + \sigma_{f,k}^n \right) \left(m_{LL}I_{L, 1}^{n+1} + m_{LR}I_{R, 1}^{n+1}\right)\frac{\Delta x_i}{6}\right] \\
    + \left[ -\frac{1}{2}\left( -F_{L,1}^{n+1}\right) + \frac{1}{2}\left( F_{L, 1}^{n+1} + F_{R, 1}^{n+1}\right)\right] =\\
    \left[\eta \chi_k \sum_{g=1}^{N_g}\sigma_{f, g}\left(m_{LL}I_{L, 1, g}^n + m_{LR}I_{R, 1, g}^n\right)\frac{\Delta x_i}{6} + \frac{1}{c\Delta t^n}\left(m_{LL}I_{L, 1}^n + m_{LR}I_{R, 1}^n\right)\frac{\Delta x_i}{6} \right]\\
    + \left[\left(m_{LL}q_{L, 1} + m_{LR}q_{R, 1}\right)\frac{\Delta x_i}{6}\right] + F^{in, +, n}
  \end{multline}
  \begin{multline}
    \left[\left(m_{LL}F_{L, 1}^{n+1} + m_{LR}F_{R, 1}^{n+1}\right)\frac{\Delta x_i}{6} + \frac{3}{4}  D_k \left( F_{L, 1}^{n+1} \right)\right]\\
    +\left[ -\frac{1}{2}D_k \left(  I_{L, 1}^{n+1}\right)  + \frac{1}{2}D_k \left(I_{L, 1}^{n+1} + I_{R, 1} ^{n+1}\right)\right]= D I^{in, +, n},
  \end{multline}
\end{subequations}


and in the right boundary cell, using \autoref{eq:right-boundary}, we have



\begin{subequations}
  \begin{multline}
    \left[\frac{1}{4} \left( I_{R, {N_x}}^{n+1}\right) + \left(\sigma_a^n + \sigma_{f,k}^n \right) \left(m_{RL}I_{L, {N_x}}^{n+1} + m_{RR}I_{R, {N_x}}^{n+1}\right)\frac{\Delta x_i}{6}\right] \\
    + \left[ \frac{1}{2}\left( F_{R, {N_x}}^{n+1} \right) - \frac{1}{2}\left( F_{L, {N_x}}^{n+1} + F_{R, {N_x}}^{n+1}\right)\right] =\\
    \left[\eta \chi_k \sum_{g=1}^{N_g}\sigma_{f, g}\left(m_{RL}I_{L, {N_x}, g}^n + m_{RR}I_{R, {N_x}, g}^n\right)\frac{\Delta x_i}{6} + \frac{1}{c\Delta t^n}\left(m_{RL}I_{L, {N_x}}^n + m_{RR}I_{R, {N_x}}^n\right)\frac{\Delta x_i}{6} \right]\\
    + \left[\left(m_{RL}q_{L, {N_x}} + m_{RR}q_{R, {N_x}}\right)\frac{\Delta x_i}{6}\right] - F^{in, -, n}
  \end{multline}
  \begin{multline}
    \left[\left(m_{RL}F_{L, {N_x}}^{n+1} + m_{RR}F_{R, {N_x}}^{n+1}\right)\frac{\Delta x_i}{6} +\frac{3}{4}  D_k \left( F_{R, {N_x}}^{n+1} \right)\right]\\
    +\left[\frac{1}{2}D_k \left( I_{R, {N_x}-1}^{n+1} \right)  - \frac{1}{2}D_k \left(I_{L, {N_x}}^{n+1} + I_{R, {N_x}} ^{n+1}\right)\right]= -DI^{in, -, n}.
  \end{multline}
\end{subequations}

The local interior equations can be represented as the block matrix

\begin{equation}
  \begin{bmatrix}
    \left(\mat{B}_{0,I} + \sigma\Delta x_i \mat{M}_w\right) & (\mat{B}_{0, F} + \mat{A})\\
    (D_{i} \mat{B}_{1,I} + D_i \mat{A}) & (\Delta x_i \mat{M}_w + D_{i} \mat{B}_{1, F})
  \end{bmatrix}\begin{bmatrix}
    I\\
    F
  \end{bmatrix} =
  \begin{bmatrix}
    S\\
    0
  \end{bmatrix}
\end{equation}

where

\begin{gather}
  \mat{B}_{0,I} = \begin{bmatrix}
    -1/4 & -1/4 & 0 & 0\\
    0 & 0 & 1/4 & 1/4
  \end{bmatrix}\\
  \mat{B}_{0, F} = \begin{bmatrix}
    -1/2 & 1/2 & 0 & 0\\
    0 & 0 & 1/2 & -1/2
  \end{bmatrix}\\
  \mat{B}_{1,I} = \begin{bmatrix}
    -1/2 & -1/2 & 0 & 0\\
    0 & -0 & 1/2 & 1/2
  \end{bmatrix}\\
  \mat{B}_{1,F} = \begin{bmatrix}
    -3/4 & 3/4 & 0 & 0\\
    0 & 0 & 3/4 & -3/4
  \end{bmatrix}\\
  \mat{A} = \begin{bmatrix}
    0 & 1/2 & 1/2 & 0 \\
    0 & -1/2 & -1/2 & 0
  \end{bmatrix}
  \mat{M}_w = \begin{bmatrix}
    0 & m_{LL} & m_{LR} & 0 \\
    0 & m_{RL} & m_{RR} & 0 
  \end{bmatrix}
\end{gather}

\begin{equation}
  I = \begin{bmatrix}
    I_{R, i-1}\\
    I_{L, i}\\
    I_{R, i}\\
    I_{L, i+1}
  \end{bmatrix}
  F = \begin{bmatrix}
    F_{R, i-1}\\
    F_{L, i}\\
    F_{R, i}\\
    F_{L, i+1}
  \end{bmatrix}
\end{equation}

\begin{equation}
  S = \begin{bmatrix}
    S_L\\
    S_R
  \end{bmatrix}.
\end{equation}

On the left boundary, 

\begin{equation}
  \begin{bmatrix}
    \left(\mat{B}_{0,I}^{L} + \sigma\Delta x_i \mat{M}^{L}\right) & (\mat{B}_{0, F}^{L} + \mat{A}_F^{L})\\
    (D_{i} \mat{B}_{1,I}^{L} + \mat{A}^{L}) & (\Delta x_i \mat{M}^{L} + D_{i} \mat{B}_{1, F}^{L})
  \end{bmatrix}\begin{bmatrix}
    I^{L}\\
    F^{L}
  \end{bmatrix} =
  \begin{bmatrix}
    S_I^{L}\\
    S_F^{L}
  \end{bmatrix}
\end{equation}

and on the right boundary, 


\begin{equation}
  \begin{bmatrix}
    \left(\mat{B}_{0,I}^{R} + \sigma\Delta x_i \mat{M}^{R}\right) & (\mat{B}_{0, F}^{R} + \mat{A}_F^{R})\\
    (D_{i} \mat{B}_{1,I}^{R} + D_i \mat{A}^R) & (\Delta x_i \mat{M}^{R} + D_{i} \mat{B}_{1, F}^{R})
  \end{bmatrix}\begin{bmatrix}
    I^{R}\\
    F^{R}
  \end{bmatrix} =
  \begin{bmatrix}
    S_I^{R}\\
    S_F^{R}
  \end{bmatrix},
\end{equation}

where, for each matrix, $^L$ denotes the last three columns only, $^R$ denotes the first three columns only, and

\begin{gather}
  I^L = \begin{bmatrix}
    I_{L, 1}\\
    I_{R, 1}\\
    I_{L, 2}
  \end{bmatrix} \quad
  F^L = \begin{bmatrix}
    F_{L, 1}\\
    F_{R, 1}\\
    F_{L, 2}
  \end{bmatrix}\\
  I^R = \begin{bmatrix}
    I_{R, N_x -1}\\
    I_{L, N_x}\\
    I_{R, N_x}
  \end{bmatrix} \quad
  F^R = \begin{bmatrix}
    F_{R, N_x -1}\\
    F_{L, N_x}\\
    F_{R, N_x}
  \end{bmatrix}\\
  S_I^L = \begin{bmatrix}
    S_{1, L} + F^{in, +, n}\\
    S_{1, R}
  \end{bmatrix} \quad
  S_F^L = \begin{bmatrix}
    D_1I^{in, +, n}\\
    0
  \end{bmatrix}\\
  S_I^R = \begin{bmatrix}
    S_{N_x, L} \\
    S_{N_X, R} - F^{in, -, n}
  \end{bmatrix} \quad
  S_F^R = \begin{bmatrix}
    0\\
    - D_{N_X}I^{in, -, n}
  \end{bmatrix}
\end{gather}

\section{Next section}


t~\cite{fleck1971implicit}

\bibliographystyle{ieeetr}
\bibliography{references.bib}


\end{document}
